% Options for packages loaded elsewhere
\PassOptionsToPackage{unicode}{hyperref}
\PassOptionsToPackage{hyphens}{url}
%
\documentclass[
  11pt,
  ignorenonframetext,
  twocolumn]{beamer}
\usepackage{pgfpages}
\setbeamertemplate{caption}[numbered]
\setbeamertemplate{caption label separator}{: }
\setbeamercolor{caption name}{fg=normal text.fg}
\beamertemplatenavigationsymbolsempty
% Prevent slide breaks in the middle of a paragraph
\widowpenalties 1 10000
\raggedbottom
\setbeamertemplate{part page}{
  \centering
  \begin{beamercolorbox}[sep=16pt,center]{part title}
    \usebeamerfont{part title}\insertpart\par
  \end{beamercolorbox}
}
\setbeamertemplate{section page}{
  \centering
  \begin{beamercolorbox}[sep=12pt,center]{part title}
    \usebeamerfont{section title}\insertsection\par
  \end{beamercolorbox}
}
\setbeamertemplate{subsection page}{
  \centering
  \begin{beamercolorbox}[sep=8pt,center]{part title}
    \usebeamerfont{subsection title}\insertsubsection\par
  \end{beamercolorbox}
}
\AtBeginPart{
  \frame{\partpage}
}
\AtBeginSection{
  \ifbibliography
  \else
    \frame{\sectionpage}
  \fi
}
\AtBeginSubsection{
  \frame{\subsectionpage}
}
\usepackage{amsmath,amssymb}
\usepackage{lmodern}
\usepackage{ifxetex,ifluatex}
\ifnum 0\ifxetex 1\fi\ifluatex 1\fi=0 % if pdftex
  \usepackage[T1]{fontenc}
  \usepackage[utf8]{inputenc}
  \usepackage{textcomp} % provide euro and other symbols
\else % if luatex or xetex
  \usepackage{unicode-math}
  \defaultfontfeatures{Scale=MatchLowercase}
  \defaultfontfeatures[\rmfamily]{Ligatures=TeX,Scale=1}
\fi
\usetheme[]{Madrid}
\usecolortheme{seahorse}
% Use upquote if available, for straight quotes in verbatim environments
\IfFileExists{upquote.sty}{\usepackage{upquote}}{}
\IfFileExists{microtype.sty}{% use microtype if available
  \usepackage[]{microtype}
  \UseMicrotypeSet[protrusion]{basicmath} % disable protrusion for tt fonts
}{}
\makeatletter
\@ifundefined{KOMAClassName}{% if non-KOMA class
  \IfFileExists{parskip.sty}{%
    \usepackage{parskip}
  }{% else
    \setlength{\parindent}{0pt}
    \setlength{\parskip}{6pt plus 2pt minus 1pt}}
}{% if KOMA class
  \KOMAoptions{parskip=half}}
\makeatother
\usepackage{xcolor}
\IfFileExists{xurl.sty}{\usepackage{xurl}}{} % add URL line breaks if available
\IfFileExists{bookmark.sty}{\usepackage{bookmark}}{\usepackage{hyperref}}
\hypersetup{
  pdftitle={Graphing Stuff},
  pdfauthor={Stats Central},
  hidelinks,
  pdfcreator={LaTeX via pandoc}}
\urlstyle{same} % disable monospaced font for URLs
\newif\ifbibliography
\setlength{\emergencystretch}{3em} % prevent overfull lines
\providecommand{\tightlist}{%
  \setlength{\itemsep}{0pt}\setlength{\parskip}{0pt}}
\setcounter{secnumdepth}{-\maxdimen} % remove section numbering
\usetheme{MAdrid}
 \usepackage{caption}
 \usepackage[absolute,overlay]{textpos}
\usepackage[utf8]{inputenc}
 \usepackage{graphbox}
\usepackage{comment}


\setlength{\fboxsep}{.8em}


\logo{
\includegraphics[height=0.5cm]{figs/unswlandscape.png}}
\titlegraphic{\includegraphics[width=0.4\textwidth]{figs/logo.png}}
\definecolor{UNSWyellow}{RGB}{249,220,0} 
\definecolor{UNSWlgreen}{RGB}{204,235,201} 
\definecolor{UNSWdblue}{RGB}{75,95,203} 
\definecolor{UNSWdgreen}{RGB}{79,200,127} 
\definecolor{UNSWdred}{RGB}{242,99,82} 


\definecolor{UNSWlblue}{RGB}{198,206,240} 

\setbeamercolor{palette primary}{bg=UNSWyellow,fg=black}
\setbeamercolor{palette secondary}{bg=UNSWyellow,fg=black}
\setbeamercolor{palette tertiary}{bg=UNSWyellow,fg=black}
\setbeamercolor{palette quaternary}{bg=UNSWdgreen,fg=black}
\setbeamercolor{structure}{fg=UNSWdblue} % itemize, enumerate, etc

% \setbeamercolor{section in head/foot}{fg=UNSWdblue} 

% Sections and subsections should not get their own damn slide.
%--------------------------------------------------------------
\AtBeginPart{}
\AtBeginSection{}
\AtBeginSubsection{}
\AtBeginSubsubsection{}


\newenvironment{cols}[1][]{}{}

\newenvironment{col}[1]{\begin{minipage}{#1}\ignorespaces}{%
\end{minipage}
\ifhmode\unskip\fi
\aftergroup\useignorespacesandallpars}

\def\useignorespacesandallpars#1\ignorespaces\fi{%
#1\fi\ignorespacesandallpars}

\makeatletter
\def\ignorespacesandallpars{%
  \@ifnextchar\par
    {\expandafter\ignorespacesandallpars\@gobble}%
    {}%
}
\makeatother

\ifluatex
  \usepackage{selnolig}  % disable illegal ligatures
\fi

\title{Graphing Stuff}
\subtitle{With SPSS}
\author{Stats Central}
\date{}
\institute{UNSW}

\begin{document}
\frame{\titlepage}

\begin{frame}[fragile]{What am I talking about?}
\protect\hypertarget{what-am-i-talking-about}{}
\begin{itemize}
\tightlist
\item
  Making good graphics for publication
\item
  Basic principles of graphing
\item
  Building blocks of graphs
\item
  Examples of common types of graph
\end{itemize}
\end{frame}

\begin{frame}{Statistical Graphs}
\protect\hypertarget{statistical-graphs}{}
``\ldots{} statistical graphics are instruments to help people reason
about quantitative information''\footnote<.->{Tufte, E. R. 2001. `The
  Visual Display of Quantitative Information', p.~91}

\begin{itemize}
\tightlist
\item
  Should be illustrative
\item
  Should be helpful
\end{itemize}

Tell your story
\end{frame}

\begin{frame}{Basic elements of a plot}
\protect\hypertarget{basic-elements-of-a-plot}{}
\begin{itemize}
\tightlist
\item
  Graph panel
\item
  Scales, labels, tick marks
\item
  Plotting symbols, line types, colour
\item
  Reference lines
\item
  Legend
\item
  Captions
\end{itemize}

Maximise the ``data-ink ratio''\footnote<.->{Tufte, E. R. 2001. `The
  Visual Display of Quantitative Information', p.~96}

\begin{itemize}
\tightlist
\item
  Use ink for data, not unnecessary decoration
\end{itemize}
\end{frame}

\begin{frame}{Basic guidelines of graphing data}
\protect\hypertarget{basic-guidelines-of-graphing-data}{}
\textbf{Make the data stand out}

\begin{itemize}
\tightlist
\item
  Let the data fill the graph panel
\item
  Don't force axes to start at zero if data values are not close to zero
  (could use ``break'' marks to show axis does not start at zero)
\item
  Label the axes clearly
\item
  Use symbols, lines, colours etc. that highlight the data
\item
  Use reference lines or grids with care
\item
  Put legends outside the graph panel
\item
  Put notes in the caption or text
\item
  Make sure the graph remains clear if made smaller (e.g.~for
  publication)
\end{itemize}
\end{frame}

\end{document}
